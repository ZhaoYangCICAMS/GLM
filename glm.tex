\documentclass[10pt]{article} %See documentation for other class options
\usepackage{tabularx}
\usepackage{color}
%\usepackage{fixltx2e,fix-cm}
%\usepackage{makeidx}
%\usepackage{multicol}
\usepackage{mathtools}
\usepackage{listings}
\usepackage{amsmath}
\usepackage{amssymb}
\usepackage{hyperref}
\usepackage{cite}
\usepackage{authblk}
\usepackage{graphicx}
\usepackage{float}
\newtheorem{thm}{Theorem}
\newtheorem{lemma}{Lemma}
\newcommand{\R}{{\mathbb R}}


\begin{document}

\floatstyle{ruled}
\newfloat{program}{thp}{lop}
\floatname{program}{Program}
\newfloat{algorithm}{thp}{lop}
\floatname{algorithm}{Algorithm}

\setlength{\parindent}{0pt}
\setlength{\parskip}{0.2em}

\definecolor{verbgray}{gray}{0.9}
\definecolor{verbgray2}{gray}{0.975}
\lstset{backgroundcolor=\color{verbgray},
  frame=single,
  framerule=0pt,
  basicstyle=\ttfamily,
  keepspaces=true,
  columns=fullflexible}


\section*{Generalized linear models, abridged}

\begin{quote}
{\it This is the 2nd major revision of this document. This version derives
algorithmic details of the iteratively re-weighted least squares method
(IRWLS), and emphasizes advantages of using the singular value decomposition
(SVD) in its implementation. We added background reference material on the SVD.
From a suggestion by James Blevins, the notation was revised to bring it closer
to other references, especially~\cite{MN}.  Although nothing here is very
deep, some details are complicated by a profusion of variables and functions.
I was inspired by Lamport's great advice~\cite{lamport} to try to
carefully spell things out very simply, step by step, referring back to
previously established quantities often.  --Bryan}
\end{quote}

\begin{quote}
Generalized linear models (GLMs) are indispensable tools in the data science
toolbox. They are applicable to many real-world problems involving continuous,
yes/no, count and survival data (and more). The models themselves are intuitive
and can be used for inference and prediction.  A few very high quality free and
open source software implementations are available (in particular within
R~\cite{R}, and also ExaStat/Revolution Analytics), as are a few first-rate
commercial ones like SAS, and Stata.

This note grew out of our own desire to better understand the numerics of
generalized linear models.  We highlight aspects of GLM implementations that we
find particularly interesting. We present some reference implementations
stripped down to illuminate core ideas; often with just a few lines of code.
Our implementations are in R but are close to pseudocode and easily ported to
other languages. --Bryan and Mike
\end{quote}


\section*{Linear algebra background material}

Skip ahead to the {\bf Linear models} section if you already know all about the
singular value decomposition...  The following brief introduction closely
follows the important reference book by Golub and Van Loan, Matrix Computations
\cite{gvl}. You should read that book.

\subsection*{Orthonormal vectors and rotations}

Let $V$ be a real-valued $n\times p$ matrix, which we write as
$V\in\R^{n\times p}$.  It's sometimes useful to enumerate the
column vectors of a matrix, which we write for instance as $V=[v_1, v_2,
\ldots, v_p]$.  The column vectors of $V$ are {\it orthonormal} if and only if
$V^T V = I$, the $p\times p$ identity matrix.  For instance, the identity
matrix itself is composed of orthonormal column vectors.  When the matrix $V$
is square, that is when $n=p$, then we simply say that $V$ is an orthonormal
matrix.
The columns of orthonormal matrices form coordinate bases of $\R^n$
whose directions are orthogonal--in other words, a rotation of
the usual unit basis coordinate system. For example, consider a $2\times 2$
orthonormal matrix $V$:
\[
V = \left(\begin{array}{cc}
1/{\sqrt{2}} & -1{\sqrt{2}} \\
1/{\sqrt{2}} & 1/{\sqrt{2}}
\end{array}\right).
\]
Figure \ref{chxx_rotation} illustrates the rotation by plotting
the column vectors of $V$ along with the usual unit basis vectors.
\begin{figure}
\begin{center}
\includegraphics[width=0.5\textwidth]{rotation.pdf}
\end{center}
\caption{Coordinates from matrix $V$ with orthonormal columns (solid lines)
compared to the standard unit basis vectors (dashed). $V$ is a rotation
matrix.}
\label{chxx_rotation}
\end{figure}
Multiplying a vector by $V$ rotates the entries of the vector to the
new coordinate system, and does nothing else. In particular, the Euclidean
norm of the vector is not changed. This is a useful result so we will state
it as a Lemma:
\begin{lemma}\label{invariant}
Let $V\in\R^{n\times n}$ be an orthonormal matrix. Then
$\|Vx\|=\|x\|$ for all $x\in\R^n$.
\end{lemma}


\subsection*{The singular value decomposition}

The singular value decomposition, or SVD, plays a central role in the
analysis--and often implementation--of many computational methods involving
matrices.  If you only plan to know one matrix decomposition, this is the one
to know.  Let $X\in\R^{n \times p}$ and let $k=\min\{n, p\}$.  Then
there exist matrices $U\in\R^{n \times k}$ and $V\in\R^{p\times
k}$ with orthonormal columns $U^T U = V^T V = I$ such that
\begin{equation}\label{SVD}
U^T X V = \Sigma,
\end{equation}
where $\Sigma$ is a $k\times k$ diagonal matrix with non-negative entries
$\sigma_1 \ge \sigma_2 \ge \cdots \ge \sigma_k \ge 0$ along its
main diagonal. This is sometimes called the ``thin'' SVD.  R users will be
familiar with this as the {\tt svd(X)} function, which returns a vector of
$\sigma_i$ values instead of a full diagonal matrix $\Sigma$ but is otherwise
the same.  The singular value decomposition is not unique, but is almost so--it
is unique up to the signs of the singular vector elements.

In the case that $n > p$ it's possible to extend the matrix $U$, by adding $n -
p$ additional orthonormal columns, into a square orthonormal $n\times n$ matrix
$\bar{U}$. Similarly, when $n < p$ we can extend $V$ to a square orthonormal
$p\times p$ matrix $\bar{V}$. The extended matrices are especially useful in
analysis, and available to R users using the function invocation
{\tt{svd(X,$\phantom{,}$nu=n,$\phantom{,}$nv=p)}}.  The extended version is
sometimes called the ``full'' SVD or just the SVD in many references and
$\bar{U}^TX\bar{V}=\bar{\Sigma}$ results in an $n\times p$ rectangular diagonal
matrix with the same main diagonal entries $\sigma_1 \ge \sigma_2 \ge \cdots \ge
\sigma_k \ge 0$ as the thin version.

The columns of $\bar{U}$ are called the {\it left singular vectors} of $X$ and the
columns of $\bar{V}$ are called the {\it right singular vectors}.  The $\sigma_i$ are
called {\it singular values} of $X$.  The SVD breaks matrix vector
multiplication into three steps: rotation, scaling, then another rotation.
Consider an $n\times p$ matrix
$X$ and its product $y$ with a vector $b\in\R^p$ using the
full SVD
$y=Xb = \bar{U}\bar{\Sigma}\bar{V}^Tb$:
\begin{enumerate}
\item Let $\hat{b}=\bar{V}^T b\in\R^p$.
Since $\bar{V}$ is orthonormal, $\hat{b}$ is simply a rotation of the
vector $b$.
\item Now let $s = \bar{\Sigma}\hat{b}\in\R^n$,
which scales each entry of $\hat{b}$ by the corresponding $\sigma_i$.
\item Finally let $y=\bar{U}s$. This is just another rotation by the
orthonormal matrix $\bar{U}$.
\end{enumerate}
The SVD reveals a lot of information about the structure of the matrix $X$.
Step 2 tells us how much a vector can be scaled by $X$.  The rotations in steps
1 and 3 tell us about its range and null space.  The number of nonzero singular
values of $X$ is equal to the \emph{rank} of $X$--the dimension of the range of
$X$ (range means the set of all linear combinations of the columns of $X$
a.k.a.  the span of $X$).  The {\it condition number}
of $X$, familiar to R users as the \verb+kappa+ function and also written
$\kappa_2(X)$, is the ratio of largest and smallest singular values.  It
measures how ill-conditioned the matrix is. Computation involving highly
ill-conditioned matrices can be very sensitive to perturbations like noise or
even numerical precision.

Let $U^TXV=\Sigma$ be the ``thin'' SVD of $X$ and let
$\bar{U}\in\R^{n\times n}$ and $\bar{V}\in\R^{p\times p}$ be
their extended versions when $n > p$ or $n < p$.  Let $r$ be the index
corresponding to the smallest non-zero singular value of $X$, for instance
$\sigma_1 \ge \sigma_2 \ge \cdots \ge \sigma_r > \sigma_{r+1} = 0 = \cdots = \sigma_k$,
where $k=\min\{n,p\}$. Then $\mbox{rank}(X) = r$ and the singular vectors define the
following bases:
\begin{itemize}
\item The first $r$ columns of $U$ form a basis of the range of $X$.
\item The first $r$ columns of $V$ form a basis of the range of $X^T$.
\item If $r<p$, then the $r+1$ through $p^{\mathrm{th}}$ columns of
$\bar{V}$ form a basis of the null space of $X$.
\item If $r<n$, then the $r+1$ through $n^{\mathrm{th}}$ columns of
$\bar{U}$ form a basis of the null space of $X^T$.
\end{itemize}
Knowledge of these bases are exceedingly useful for analysis, as well as for
development and implementation of good algorithms.
See Golub and Van Loan\cite{gvl} for many more details.


Several important {\it matrix norms} can be defined in terms of the SVD.
Matrix norms provide a way to measure distance between matrices and exhibit
the usual properties of norms plus a sub-multiplicative property
$\|X Y\| \le \|X\| \|Y\|$ for conformable matrices $X$ and $Y$.

Let $X\in\R^{n\times p}$, and let $k=\min(n, p)$. Then the squared
{\it Frobenius norm} of $X$ is
$\|X\|_F^2 = \sigma_1^2 + \sigma_2^2 + \cdots + \sigma_k^2$.
The {\it spectral norm} of $X$ is simply its largest singular value,
$\|X\|_2 = \sigma_1$. Both norms are familiar to R users of the
{\tt norm} function.



The SVD gives us a practical way to think about matrix rank.  The smallest
singular value of a matrix $X$ is the spectral norm distance between $X$ to the
set of rank-deficient matrices; that is, it's a measure of how close to
rank-deficient $X$ is.

It might not make sense to think about scaling vectors by values less than the
floating-point unit roundoff $\mathbf{u}$ ({\tt .Machine\$double.eps} in R). In
this case, if the smallest singular value of $X$ is near or below $\mathbf{u}$,
then we might be in trouble--computations involving $X$ beware!  Matrices with
smallest singular values below $\mathbf{u}$ or some small multiple of
$\mathbf{u}$  are called {\it numerically rank deficient}.



\subsection*{Linear models}

Let $X\in\R^{n\times p}$ be a matrix formed from $n$ data observations of $p$
features (called a model matrix).  Let $y\in\R^n, y\ne 0$ be a vector of $n$
measurements (called a response vector) and assume that the entries of $y$
are realizations of a set of $n$ independent but identically-distributed
(i.i.d.) random variables.  Then the linear system of equations
\begin{equation}\label{linear}
Xb = y + e
\end{equation}
is what we mean when we say ``linear model,'' where a possible solution
$b\in\R^p$, when one exists, is a vector of model coefficients, and the
entries of the residual error term $e\in\R^n$ are also i.i.d. 
Given a vector $b$ of model
coefficients, we call the product $Xb$ a {\it predictor} of the response vector
$y$.  We're leaving important definitions of ``random variable'' and
``independent and identically distributed'' to the references.

It's easy to see that if $e=0$, the exactly-specified linear model Equation
\ref{linear} admits three possibilities for $b$:
\begin{enumerate}\setlength\itemsep{-0.4em}
\item No solution for $b$ exists.
\item A unique solution $b$ exists.
\item An infinite number of solutions $b$ exist.
\end{enumerate}
For instance, in the 2-D case of a two by two matrix $X$ the possible solutions
correspond to 1) parallel lines; 2) lines intersecting at a point; and
3) co-incident lines.
The interesting cases 1 and 3 correspond to situations when
$\mathrm{rank}(X) < p$.
Dealing with those interesting cases substantially complicate
solution implementations, even when $e\ne 0$ but is allwoed to consist
of i.i.d. entries.

Given a model matrix $X$ and response vector $y$, there are many ways we might
go about solving for a vector of model coefficients $b$. A particularly
natrural approach is described next.


\subsection*{Ordinary least squares}

The ordinary least squares (OLS) solution to Equation \ref{linear} 
finds a coefficient vector $b$ that
minimizes the residual error between the response and the predictor
vectors,
\[
\min_b \|X b - y\|^2,
\]
where $||\cdot||$ indicates Euclidean norm.  The conditions $p \le n$ and
$\mathrm{rank}(X)=p$ imply that there exists a unique OLS solution--such cases
are called full rank least squares problems.  Otherwise, if more that one
solution exists we can add constraints to form a new problem with a unique
solution, leading to many possible constrained solution methods.
For instance, we may constrain the solution to be the one with minimal
Euclidean norm. Other constrained solutions are possible.

Let's see how the SVD can be used to solve general ordinary least squares
problems. The following result is adapted from Golub and Van Loan
\cite[Theorem 5.5.1]{gvl}.
\begin{thm}
Let $X$ be a real $n\times p$ matrix, with full SVD
$\bar{U}^TX\bar{V} = \bar{\Sigma}$ using extended matrices
$\bar{U} = [u_1, u_2, \ldots, u_n] \in\R^{n\times n}$,
$\bar{\Sigma}\in\R^{n\times p}$, and
$\bar{V} = [v_1, v_2, \ldots, v_p]\in\R^{p\times p}$,
and let $r \le \min\{n, p\}$ be the rank of $X$. Then
\begin{equation}\label{SVDLS}
b_{LS} = \sum_{i=1}^r \frac{u_i^T y}{\sigma_i}v_i
\end{equation}
minimizes $\|X b - y\|^2$ and has the smallest Euclidean norm of all
such minimizers.
\\
{\bf Proof}. For any vector $b\in\R^p$, 
\begin{align}
\|X b - y\|^2 &= \|\bar{U}\bar{\Sigma}\bar{V}^T b - y\|^2 \qquad\,\,\,\,\,\,\, (\mbox{replacing X with its full SVD})\nonumber\\
&= \|\bar{U}^T(\bar{U}\bar{\Sigma}\bar{V}^T b - y)\|^2 \,\,\,\,\,\, (\mbox{by Lemma \ref{invariant}})\nonumber\\
&= \|\bar{\Sigma}\bar{V}^T b - \bar{U}^Ty\|^2 \nonumber \\
&= \sum_{i=1}^p(\sigma_iv_i^T b - u_i^Ty)^2 + \sum_{i=p + 1}^n(u_i^Ty)^2. \label{residual}
\end{align}
The columns of $\bar{V}$ for an orthonormal basis of $\R^p$.
Express the solution $b$ as a linear combination of the column vectors $v_i$,
$b = \sum_{i=1}^p \gamma_i v_i$ (that is, $\gamma_i = v_i^Tb$). Since $\mathrm{rank}(X)=r$ then
$\sigma_{r+1} = \sigma_{r+2} = \cdots = \sigma_p = 0$ and the corresponding
coefficients $\gamma_i$ may take on any value without affecting the
residual norm.  The specific choice
$\gamma_{r+1} = \gamma_{r+2} = \cdots b_p = 0$
minimizes the norm of any possible solution $b$.
Then the residual norm in Equation \ref{residual} is minimized by setting the remaining coefficients
$v_i^Tb = \gamma_i = (u_i^T y) v_i/ \sigma_i$ for $i=1, 2, \ldots, r$.
$\square$
\end{thm}

The minimum-norm least squares solution is widely used in many science and
engineering applications, but it's more common in statistics to constrain
solutions to rank deficient problems in other ways. In particular, we will see
below how R reformulates rank-deficient problems into full-rank ones by
selecting a subset of columns using a heuristic procedure based on the model
matrix column order. Other subset selection approaches include the lasso which
solves the penalized least-squares problem
\begin{equation}\label{lasso}
\min_b\|Xb - y\|^2 + \mu\|b\|_1
\end{equation}
for a parameter $\mu>0$ and the 1-norm of the solution vector $b$. The lasso
is the closest convex estimate of the parameterized
\emph{best subset selection problem}:
\[
\min_b\|Xb - y\|^2 + \mu\|b\|_0,
\]
where $\|b\|_0$ means simply the count of nonzero components of $b$.
(Despite the notation, $\|b\|_0$ is not a vector norm
since for any scalar $\lambda$ with $|\lambda |$ not equal to zero or one,
$\|\lambda b\|_0 \ne |\lambda |\|b\|_0$.) Although the best subset selection
problem might seem to be the most natural way to select subsets of columns of
the matrix $X$, the problem is nonconvex and hard to solve--indeed it is
known to be NP hard.  We
shall see later that there are other approaches to estimating optimal
column subsets including a
fast heuristic method by Golub called SVD subset
selection~\cite[Section 12.2]{gvl}, and a
newer approach by Lanza, Reichel and others based on Krylov subspace
methods~\cite{lanza}.

The ordinary least squares solution of linear models has important statistical
properties shown by Gauss~\cite{gauss} and later rediscovered by
Markoff~\cite{markoff}. The least squares solution defines a {\it minimum
variance unbiased estimator}, the technical details of which we leave to the
references, in particular see~\cite{hastie},\cite{MN}.


\section*{Generalized linear models}

Our notes on generalized linear models closely follow the seminal book
``Generalized Linear Models'' by McCullagh and Nelder~\cite{MN}.  That very
readable and practical book remains, in our opinion, the best all-around applied
reference on GLMs and strongly influenced algorithm implementations in the
R language.
McCullagh and Nelder describe generalizations of the basic linear model
in three parts:
\begin{enumerate}
\item A \emph{random component} describing the distribution of the
measured entries of of a response vector $y$ and their vector of
expected values $\mu = E(y)\in\R^n$.
\item A \emph{systematic component} $\eta = X\beta$ that is just a basic
linear model involving a vector $\eta\in\R^n$, model matrix $X\in\R^{n\times p}$ and
coefficient solution vector $\beta\in\R^p$.
\item A \emph{link function} between the random and systematic components,
$\eta = g(\mu)$, applied component-wise to the vector $\mu$.
\end{enumerate}
The link function $g$ is assumed to be a real-valued monotonic, differentiable
(and therefore invertible) function.  These notes assume that the random
component distribution describing the response belongs to a one- or
two-parameter \emph{exponential family} of probability distributions described
below. The exponential family covers many widely used and important cases like
logistic/binomial, Bernoulli, multinomial, exponential, Poisson, Gaussian, and
others. Limiting our discussion to models that fit into the exponential family,
despite a superficial mathematical complexity, greatly simplifies many
details.

Adding the random component and link function around a basic linear model
allows generalized linear models to model a wider range of scenarios. In
particular, the link function lets us model variables $\mu$ that are restricted
to intervals, for instance the interval $[0,1]$ useful for modeling binary
values.  And we can use the random component to pair an appropriate
distribution with such values (say, a binomial distribution in the case of 0/1
data). The added modeling flexibility comes with a cost--the link function can
turn finding the solution to GLMs into a nonlinear problem (despite the
underlying linear model assumption in the systematic component).


\subsection*{The exponential family of distributions}

The following sections include a lot of math notation and many functions and
parameters to keep track of. Although a bit complicated, nothing presented here
is harder than elementary Calculus. We closely follow McCullagh and Nelder's
exposition~\cite{MN}, but we expand on it in some places to help illuminate key
ideas.

The exponential family of distributions can be written as a function
\begin{equation}\label{expfamily}
f(y; \theta, \phi) = \exp\left(\frac{y\theta - b(\theta)}{a(\phi)} + c(y, \phi)\right),
\end{equation}
for parameters $\theta$ and $\phi$ and fixed functions $a,b,$ and $c$.
The notation $f(y; \theta, \phi)$ means a function $f(y)$ that
depends on the given parameters $\theta$ and $\phi$.
Any probability distribution that can be re-written in this form belongs
to the exponential family.

For instance, let
$\theta=\mu$, $\phi=\sigma^2$, $b(\theta)=\theta^2/2$, $a(\phi)=\phi$
and $c(y, \phi) = -\frac{1}{2}(\frac{y^2}{\sigma^2} + \log(2\pi\sigma^2))$.
Then substituting those values in yields
\begin{align*}
f(y; \theta, \phi) &=
  \exp\left(\frac{y\mu - \mu^2/2}{\sigma^2} - \frac{1}{2}\left(\frac{y^2}{\sigma^2} + \log(2\pi\sigma^2)\right)\right)\\
&= \frac{1}{\sqrt{2\pi\sigma^2}}\exp\left(\frac{-(y - \mu)^2}{2\sigma^2}\right),
\end{align*}
which is a standard expression of a normal distribution,
showing that the normal distribution fits in to the exponential
family.

Similarly, consider the Poisson distribution with single parameter
$\mu$,
\[
\exp(-\mu)\mu^y/{y!}\,\,.
\]
Let $\theta=\log\mu$, $a(\phi)=1$, $b(\theta)=\exp\theta$, and
$c(y, \phi)=-\log{(y!)}$. Then
\begin{align*}
f(y; \theta, \phi) &=  \exp\left(\frac{y\theta - b(\theta)}{a(\phi)} + c(y, \phi)\right)\\
&= \exp\left(\frac{y\log\mu - \exp\theta}{1} - \log y!\right)\\
&= \frac{\exp(y\log\mu - \exp\theta)}{y!}\\
&= \frac{\exp(y\log\mu - \exp\log\mu)}{y!}\\
&= \exp(y\log\mu - \mu)/y!\\
&= \exp{(-\mu)}\exp{(y\log\mu)}/y!\\
&= \exp(-\mu)\mu^y/y!\,\,.
\end{align*}
Many other distributions are described by the exponential family and can be
written in the form of Equation \ref{expfamily}, including discrete
and continuous distributions.

\subsubsection*{The log-likelihood function}
The function
\begin{equation}\label{loglik}
\ell(\theta, \phi; y) = \frac{y\theta - b(\theta)}{a(\phi)} + c(y, \theta)
\end{equation}
is the corresponding \emph{log-likelihood} function. In this context the
function $b(\theta)$ is called the \emph{cumulant function} and $\phi$
the \emph{dispersion parameter}. The log-likelihood function
indicates how likely a given observed value $y$ is as a function
of parameters $\theta$ and $\phi$.  Next we derive a few basic identities that
will be useful later.

Let $\partial\ell/\partial\theta$ be the derivative of the log-likelihood
function with respect to $\theta$ (how much the function changes as
$\theta$ changes), and similarly $\partial^2\ell/\partial\theta^2$
its 2nd derivative (how much the derivative function changes as
$\theta$ changes). Then
\begin{equation}\label{dl}
\frac{\partial\ell}{\partial\theta} = \frac{y - b'(\theta)}{a(\phi)},
\end{equation}
and,
\begin{equation}\label{d2l}
\frac{\partial^2\ell}{\partial\theta^2} = \frac{-b''(\theta)}{a(\phi)},
\end{equation}
where $b'(\theta)$ means the derivative of the function $b$ taken with
respect to $\theta$.

Assume that $a(\phi)\ne 0$ and that 
the expected value $E(\partial\ell/\partial\theta) = 0$
and also that
$E(\partial^2\ell/\partial\theta^2) + E(\partial\ell/\partial\theta)^2 = 0$.
Then
\begin{equation}\label{bprime}
0 = E(\partial\ell/\partial\theta)  = \frac{E(y) - b'(\theta)}{a(\phi)}
\qquad\mbox{which means that}\,\,b'(\theta) = E(y).
\end{equation}
Recall above that we sometimes use the alternative notation $\mu=E(y)$
for the expected value of $y$; so $\mu=E(y)=b'(\theta)$.

Similarly,
\begin{align}
0 &= 
E(\partial^2\ell/\partial\theta^2) + E(\partial\ell/\partial\theta)^2 \nonumber\\
&=
\frac{-b''(\theta)}{a(\phi)} +
E\left(\frac{y - b'(\theta)}{a(\phi)}\right)^2 \nonumber\\
&= \frac{-b''(\theta)}{a(\phi)} +
E\left(\frac{y^2 - 2yb'(\theta) + b'(\theta)^2}{a(\phi)^2}\right)\nonumber\\
&= \frac{-b''(\theta)}{a(\phi)} +
\frac{E(y^2) - E(y)^2}{a(\phi)^2} \qquad(\mbox{substituting}\,\,b'(\theta)=E(y))\nonumber\\
&= \frac{-b''(\theta)}{a(\phi)} + \frac{V(y)}{a(\phi)^2}\nonumber\\
&\mbox{which means that}\,\,a(\phi)b''(\theta) = V(y)\label{b2},
\end{align}
where $V(y)$ is the usual definition of the variance function for $y$.

Finally for this section, one more useful identity showing that the
rate of change of the expected value of $y$ with respect to the parameter $\theta$
is a multiple of the variance function $V(y)$:
\begin{align}
\frac{d}{d\theta}E(y) &= \frac{d}{d\theta}\mu &\mbox{(just notation)}\nonumber\\
&= \frac{d}{d\theta}b'(\theta) &\mbox{(by Equation \ref{bprime})}\nonumber\\
&= b''(\theta)\nonumber\\
&= V(y)/a(\phi)\label{mutheta} &\mbox{(by Equation \ref{b2})}.
\end{align}

\subsubsection*{Canonical link functions}

An important subset of the exponential family is specified by particular
choices of link functions called \emph{canonical link functions}.  Recall that
the link function relates $\eta$ and $\mu$ by $\eta=g(\mu)$ in a generalized
linear model, and thus $d\eta/d\mu = g'(\mu)$. Choosing a canonical link
results in a number of simplifications as we will see. Chief among them for our
purposes, a canonical link connects $d\eta/d\mu$ to the variance function by
\begin{equation}
\label{canonical}
{d\eta}/{d\mu}={1}/{V(\mu)}.\qquad\mbox{(canonical link case)}
\end{equation}



\subsection*{GLMs and the exponential family}

The last section introduced the random component of generalized linear models
and corresponding log-likelihood function for the exponential family of
distributions.  This section puts that together with the remaining
generalizations, the systematic component's linear model $\eta = X\beta$ and
the link function $\eta=g(\mu)$ (where, again, $\mu$ is simply an alternative
notation for the expected value of the response vector $E(y)$).

One approach for solving generalized linear models is to find the value of the
coefficient vector $\beta$ that maximizes the log-likelihood function. Solving
for such a \emph{maximum-likelihood solution} is the main goal of this section.
In order to find a maximum using Calculus, we will need (at least) an
expression for the derivative of the log-likelihood function with respect to
each component of the solution $\beta_j$, $\partial{l}/\partial{\beta_j}$.

Writing the $n \times p$ matrix $X$ showing each column as
$X = [x_1, x_2, \cdots, x_p]$, then
$\eta = X\beta = x_1\beta_1 + x_2\beta_2 + \cdots + x_p\beta_p$,
and 
\begin{equation}\label{eta}
\partial\eta/\partial\beta_j = x_j.
\end{equation}

Then using the chain rule from Calculus, the derivative of the log-likelihood
function with respect to each component of the solution $\beta_j$ is
\begin{align}
\frac{\partial\ell}{\partial\beta_j}
&= \frac{\partial\ell}{\partial\theta}\frac{d\theta}{d\mu}\frac{d\mu}{d\eta}\frac{\partial\eta}{\partial\beta_j} \nonumber\\
&= \frac{\partial\ell}{\partial\theta}\frac{1}{V(\mu)}\frac{d\mu}{d\eta}\frac{\partial\eta}{\partial\beta_j} 
    &\mbox{(by Equation \ref{mutheta})} \nonumber\\
&= \frac{\partial\ell}{\partial\theta}\frac{1}{V(\mu)}\frac{d\mu}{d\eta}x_j
    &\mbox{(by Equation \ref{eta})} \nonumber\\
&= \frac{y- b'(\theta)}{a(\phi)}\frac{1}{V(\mu)}\frac{d\mu}{d\eta}x_j
    &\mbox{(by Equation \ref{dl})} \nonumber\\
&= w\frac{y-\mu}{a(\phi)}\frac{d\eta}{d\mu}x_j,
    &\mbox{(by Equations \ref{bprime} and \ref{w})} \label{dldb}
\end{align}
where $w$ is a vector with $n$ elements defined as a multiple of the inverse variance function
(where division is carried out component-wise):
\begin{equation}\label{w}
w = \left(\frac{d\mu}{d\eta}\right)^2 \bigg/ V(\mu).
\end{equation}
We remark that when $g$ is a canonical link function we get a simplification
for $w$ using~\ref{canonical}. In the canonical link case,
\begin{align}
w &= \left(\frac{d\mu}{d\eta}\right)^2 \bigg/ V(\mu) \nonumber \\
&= V^2(\mu) / V(\mu) \nonumber \\
&= V(\mu)\qquad\mbox{(canonical link case)}.  \label{W_canonical}
\end{align}
Assume that the dispersion function $a(\phi)$ is constant with respect to the
solution $\beta$. Then the maximum of the log-likelihood function $\ell$ with
respect to each solution component $\beta_j$ for $j=1, 2, \ldots, p$ occurs when
\begin{equation}\label{max_loglik}
\sum_{i=1}^n w_i(y-\mu)_i\frac{d}{d\mu}\eta_ix_{ij} = 0.
\end{equation}
Since $\eta=g(\mu)$, the term $\frac{d}{d\mu}\eta$ is
simply $g'$, the derivative of the link function.
Assemble entries of $w$ and $g'$ along the diagonal of an
$n\times n$ diagonal matrix $W$:
\begin{equation}\label{W}
W_{ij} = \bigg\{\begin{array}{cr}
w_i / g'(\mu)_i & \mbox{if $i=j$}, \\
0 & \mbox{otherwise}.
\end{array}
\end{equation}
Then a compact formula for the gradient of $\ell$ with respect to $\beta$ is
\begin{equation}\label{gradient}
\nabla_\beta\ell = XW(y - \mu).
\end{equation}

At this point, we have enough information to formulate a first-order solution
method to finding the maximum likelihood GLM solution.  Possible solution
methods include gradient descent, Gauss-Newton, conjugate gradient, or
a quasi-Newton approach. The following example implementation uses R's
\verb+optim+ function to make a simple GLM solver using the BFGS quasi-Newton
algorithm.

XXX EXAMPLE XXX



\subsection*{GLMs solved using Newton's method}

We can do better than the quasi-Newton solution in the last section.  Because
we restricted our problems to the exponential family, we can formulate an
analytic representation of the second derivatives of the log-likelihood
function to form a Hessian matrix. That knowledge enables us to employ solution
methods using second-order (quadratic) approximations like Newton's
method--such methods have more favorable convergence properties (faster, more
stable) than first order solution methods.

We need to differentiate the expression for the derivative of $\ell$ in
Equation~\ref{dldb} with respect to other elements
of $\beta_k$ to compute the entries of the 2nd derivative Hessian matrix,
and again we assume that the dispersion term $a(\phi)$ is constant
with respect to the solution $\beta$. Then the $jk$-entry of the
Hessian matrix is:
\begin{align}
H_{jk} &= \frac{\delta^2\ell}{\delta\beta_k\delta\beta_j}
 = \frac{\delta}{\beta_k}\frac{\delta\ell}{\delta\beta_j} \nonumber\\
&=
\frac{\delta}{\delta\beta_k}
\sum_i\left(
w_i(y-\mu)_i\frac{d\eta_i}{d\mu}x_{ij} \right)\nonumber \\
&= \sum_i\left(
(y-\mu)_i
\frac{\delta}{\delta\beta_k}
w\frac{d\eta_i}{d\mu}x_{ij}
+
w_i\frac{d\eta_i}{d\mu}x_{ij}\frac{\delta}{\delta\beta_k}(y-\mu)_i \label{hsum}
\right)
\end{align}
where the last equality uses the product rule from Calculus.
At this point, we have an (unwieldy) expression for the Hessian and
we can plug that together with the gradient function from Equation~\ref{gradient}
into Newton's method to get a maximum likelihood GLM solver.
However, we will consider an important special case next that is much simpler.

\subsubsection*{Canonical link case}
In the case of a canonical link, the expression $w_i\frac{d\eta_i}{d\mu}$ in the
first term of the sum in Equation~\ref{hsum}
is constant because in such cases
$d\eta/d\mu=1/V(\mu)$ and
$w=V(\mu)$ 
by Equations~\ref{canonical}i and~\ref{W_canonical}. Thus, its derivative
\[
\frac{\delta}{\delta\beta_k} w_i\frac{d\eta_i}{d\mu} = 0,
\]
that is, the first term of the Hessian in Equation~\ref{hsum}
drops out when $g$ is a canonical link function.
Meanwhile, consider the second term
\begin{align*}
\sum_i\left(
w_i\frac{d\eta_i}{d\mu}x_{ij}\frac{\delta}{\delta\beta_k}(y-\mu)_i
\right)  &=
\sum_i\left(
w_i\frac{d\eta_i}{d\mu}x_{ij}\frac{\delta\mu}{\delta\beta_k} \right) &\mbox{($y$ is constant wrt $\beta$)}  \\
&= 
\sum_i\left(
w_ix_{ij}\frac{\delta\eta_i}{\delta\beta_k} \right)&\mbox{(chain rule)}\\
&= 
\sum_i\left(
w_ix_{ij}x_{ik} \right), &\mbox{(by Equation \ref{eta})}
\end{align*}
finally arriving (with a substantial subscript-induced headache) at a compact
expression for the Hessian using the definition of $W$ from Equation~\ref{W}:
\begin{equation}\label{Hessian}
H = X^T W X.
\end{equation}

With an expression for the gradient from Equation~\ref{gradient} and Hessian
from Equation~\ref{Hessian} of the log-likelihood function, we have all we need
to implement a second order solution method. The next example uses R's
\verb+nlm+ function to find the maximum likelihood GLM solution using Newton's
method.

XXX EXAMPLE XXX

Remember that this derivation assumed that $g$ is a canonical link function. In
the general case we need to resort to the definition of the Hession in
Equation~\ref{hsum} for a Newton's method-based solution.



\subsection*{Iteratively re-weighted least squares}




The numerical solution of model problems of this form was carefully analyzed by
Paige\cite{paige}.
...entries of $W$ are non-zero, the generalized linear model
\ref{glm} results in a weighted nonlinear least squares problem
typically solved by the iteratively reweighted least square method
shown in Algorithm \ref{irls} and defined carefully by Bj\"ork\cite{bjork}...

\subsection*{Numerical implementation issues}

cover edge cases here including zero-variance observations (constant rows in $X$)
and singular/ill-conditioned $X$.

Introduce R's rank revealing QR irwls approach.

SVD-irwls based on algorithm by O'Leary

comparison/examples between R's RRQR-irwls and SVD-irwls

\subsubsection*{Round off error in QR- and SVD-based methods}











\section*{Copyright}
Copyright \copyright  2014 Michael Kane and Bryan W. Lewis

\begin{quote}
    Permission is granted to copy, distribute and/or modify this document
    under the terms of the GNU Free Documentation License, Version 1.3
    or any later version published by the Free Software Foundation;
    with no Invariant Sections, no Front-Cover Texts, and no Back-Cover Texts.
    A copy of the license is included in the Github project files.
\end{quote}



\begin{thebibliography}{99}
\bibitem{anda} Anda, A. and Park, H., Self-scaling fast rotations for stiff least squares problems, Lin. Alg. Appl., 234, 1996, pp. 137-162.
\bibitem{bjork} Bj\"orck, \AA.,  Numerical Methods for Least Squares Problems, SIAM, Philadelphia, 1996.
\bibitem{bates} Bates, D., \url{http://www.stat.wisc.edu/courses/st849-bates/lectures/GLMH.pdf">http://www.stat.wisc.edu/courses/st849-bates/lectures/GLMH.pdf}.
\bibitem{dekker}
Dekker, Theodorus Jozef. ``A floating-point technique for extending the available precision.'' Numerische Mathematik 18.3 (1971): 224-242.
\bibitem{friedman} Friedman, Jerome, Trevor Hastie, and Rob Tibshirani. "Regularization paths for generalized linear models via coordinate descent." Journal of statistical software 33.1 (2010): 1.
\bibitem{glmnet} Friedman, Hastie, Tibshirani, Simon, Narasimhan, Qian, \url{https://cran.r-project.org/package=glmnet}.
\bibitem{gauss} Gauss, C. F., Theoria combinationis observationum erroribus minimis obnoxiae, Pars prior, 1863 (1st written 1821).
\bibitem{hastie} Hastie, T. J. and Pregibon, D., Generalized linear models, Chapter 6 of Statistical Models in S, eds J. M. Chambers and T.  J. Hastie, Wadsworth \& Brooks/Cole, 1992.
\bibitem{fmm} Forsythe, George Elmer, Cleve B. Moler, and Michael A. Malcolm. ``Computer methods for mathematical computations.'' (1977).
\bibitem{gvl} Golub, Gene H., and Charles F. Van Loan. Matrix computations. Vol. 3. JHU Press, 2012.
\bibitem{higham96} Higham, Nicholas J. Accuracy and stability of numerical algorithms. Vol. 80. Siam, 2002.
\bibitem{horn-johnson} Horn, Roger A., and Charles R. Johnson. Matrix analysis. Cambridge university press, 1990.
\bibitem{jordan} Jordan, Michael, \url{https://people.eecs.berkeley.edu/~jordan/courses/260-spring10/other-readings/chapter8.pdf} (online notes).
\bibitem{lamport} Lamport, Leslie. How to write a proof. The American mathematical monthly 102.7 (1995): 600-608.
\bibitem{lanza}Lanza, Alessandro, et al. A Generalized Krylov Subspace Method for $\ell_p-\ell_q$ Minimization. SIAM Journal on Scientific Computing 37.5 (2015): S30-S50.
\bibitem{lumley} Lumley, T, \url{http://cran.r-project.org/web/packages/biglm">http://cran.r-project.org/web/packages/biglm}.
\bibitem{markoff} Markoff, A., Wahrscheinlichheitsrechnug,  Leipzig, 1912.
\bibitem{MN} McCullagh P. and Nelder, J. A., Generalized Linear Models, Chapman and Hall, London 1989.
\bibitem{oleary} O'Leary, D., Robust regression computation using iteratively reweighted least squares, Siam J. Mat. Anal. Appl., Vol. 11 No. 3, 1990, pp. 466-480.
\bibitem{paige} Paige, C. C., Fast numerically stable computations for generalized least squares problems, Siam J. Num. Anal., 16, 1979, pp. 165-171.
\bibitem{R} The R project \url{http://www.r-project.org">http://www.r-project.org}.
\bibitem{trefbau} Trefethen, Lloyd N., and David Bau III. Numerical linear algebra. Vol. 50. Siam, 1997.
\bibitem{zhou} Zhou, H. and Hastie, T., Regularization and Variable Selection via the Elastic Net, J. Royal Statistical Society, B, 2005, pp. 301-320.
\end{thebibliography}


\end{document}
